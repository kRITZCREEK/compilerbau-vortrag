\documentclass[12pt,german]{article}
\usepackage[german]{babel}
\usepackage[utf8]{inputenc}
\usepackage[T1]{fontenc}

\usepackage{hyperref}
\usepackage{natbib}
\usepackage{xcolor}

%%%%%%%%%%%%%%%%% CodeBlocks %%%%%%%%%%%%%%%%%%%%%%
\usepackage{listings}
\lstset{language=Haskell,
        basicstyle=\footnotesize\ttfamily
        ,frame=single
        ,backgroundcolor=\color{lightgray}
      }

%%%%%%%%%%%%%%%%% \CodeBlocks %%%%%%%%%%%%%%%%%%%%%%

%%%%%%%%%%%%%%%%% Title %%%%%%%%%%%%%%%%%%%%%%
\author{Christoph Hegemann}
\title{LangLang\\Monadische Kombinatoren in Haskell}
\date{\today}
%%%%%%%%%%%%%%%%% \Title %%%%%%%%%%%%%%%%%%%%%%

%%%%%%%%%%%%%%%%% HeaderFooter  %%%%%%%%%%%%%%%%%%%%%%
\usepackage{fancyhdr}
\pagestyle{headings}
%%%%%%%%%%%%%%%%% \HeaderFooter %%%%%%%%%%%%%%%%%%%%%%

\begin{document}

\maketitle
\newpage
\tableofcontents

\section{Einleitung}
In dieser Dokumentation werde ich den Entwicklungsprozess
sowie die getroffenen Enscheidungen bei der Durchführung des
Compilerbauprojektes darlegen. Zunächst werde ich einige vorab
getroffene Entscheidungen erklären.
\\

Meine Sprache sollte eine Interpretierte sein, dies liegt vor allem an
meiner eigenen Vorliebe für REPLs und meiner Ansicht das eine Sprache ohne
eben jene nicht angehm für die Software Entwicklung zu verwenden ist.
\\

\noindent
Als Sprache habe ich mich aus folgenden Gründen für \textbf{Haskell}
entschieden:
\begin{itemize}
\item Deklarativer Stil
\item Interesse an der Sprache(rein funktional)
\item Parsec Bibliothek
\end{itemize}

Zum Thema Parsec und der besonderen Unterstützung von Haskell für
diese komme ich später noch einmal genauer. Vorab gesagt sei aber das
durch die Verwendung eben jener für das Lexen und Parsen keine einzige
Zeile Code generiert werden muss.

\section{Spezifikation Lang}
\subsection{Datentypen}

Die Datentypen sind im Projekt unter \textit{Langlang/Data.hs} zu finden.\\
Der deklarative Stil von Haskell, sowie das Vorhandensein von
algebraischen Datentypen erlaubt es hier die Primitve
aufzulisten, indem der Code gezeigt wird der diese auch definiert:

\begin{lstlisting}
data Ausdruck = Konstante Double
                | Bezeichner String
                | Addition Ausdruck Ausdruck
                | Subtraktion Ausdruck Ausdruck
                | Multiplikation Ausdruck Ausdruck
                | Division Ausdruck Ausdruck
                | Modulo Ausdruck Ausdruck
                | Negation Ausdruck
                | FunktionsAufruf String Ausdruck
                | Konditional Bedingung Ausdruck Ausdruck

data Bedingung = Und Bedingung Bedingung
               | Oder Bedingung Bedingung
               | Nicht Bedingung
               | Gleich Ausdruck Ausdruck
               | Kleiner Ausdruck Ausdruck
               | KleinerGleich Ausdruck Ausdruck

data Anweisung = Ausgabe Ausdruck
               | Zuweisung String Ausdruck
               | FunktionsDefinition String FunktionsKoerper

data FunktionsKoerper = FunktionsKoerper String Ausdruck
\end{lstlisting}

Dies sieht noch sehr nach der Sprachdefinition im Stile der LL(1)
Parser aus. Im nächsten Zug werde ich mich der Parsertätigkeit
widmen und demonstrieren wie Monadische Kombinatoren den rekursiven
Abstieg leicht aussehen lassen.

\section{Parsing}

Die Parser sind im Projekt unter \textit{Langlang/Parsers.hs} zu
finden. Als mächtige Bibliothek hat kann Parsec lexer für Operatoren
und reservierte Wörter anlegen.

\begin{lstlisting}
lexer :: TokenParser ()
lexer = makeTokenParser
        (javaStyle { opStart  = oneOf "+-*/%|&=!<>"
                   , opLetter = oneOf "+-*/%|&=!<>"
                   , reservedNames = ["let", "def", "print"
                                     ,"if", "then", "else"]})
\end{lstlisting}

\noindent Dies hat den Vorteil das wir uns direkt mit den interessanten Fällen
beschäftigen können.

\noindent Hier einmal der  Nummernparser zeilenweise als Beispiel:
\begin{itemize}
\item \texttt{parseNummer = do}\\
Defintion sowie eintreten in den do-Block $\rightarrow$ Wir haben es mit einer
monadischen Funktion zu tun.
\item \texttt{wert $\leftarrow$ naturalOrFloat lexer}\\
Der vorhin definierte Lexer konsumiert alles was er für einen
Int/Float hält und legt dies unter wert ``ab''.
\item \texttt{case wert of}\\
Eine der viele Stärken von Haskell. PatternMatching gegen den
(algebraischen!) Returntyp von naturalOrFloat.
\item \texttt{Left i $\rightarrow$ return \$ Konstante \$ fromIntegral i}\\
Der Lexer hat einen Int geparsed. Da wir nur mit Floats arbeiten
müssen wir mit \textit{fromIntegral} die Zahl konvertieren.
\item \texttt{Right n $\rightarrow$ return \$ Konstante \$ n}\\
Der Wert ist schon ein Float.
\end{itemize}

Nachdem wir nun einen Recht trivialen Parser gesehen haben kommt nun
der interessante ``Kombinator'' twist von Parsec.
Nehmen wir als Beispiel die die Definition für den sehr abstrakten
Begriff Term:
\begin{lstlisting}
parseTerm :: Parser Ausdruck
parseTerm =
  parens lexer parseAusdruck
  <|> parseNummer
  <|> try parseFunktionsAufruf
  <|> try parseKonditional
  <|> (identifier lexer >>= return . Bezeichner)
\end{lstlisting}

Dies lässt sich wie folgt lesen:
\begin{enumerate}
\item Ein Term ist entweder ein geklammerter Ausdruck
\item oder eine Zahl
\item oder ein Funktionsaufruf
\item oder ein Konditional
\item oder eine Zuweisung
\end{enumerate}

\noindent Wobei für die Prüfung auf Funktionsaufruf und Konditional Backtracking
eingesetzt werden muss falls ein Parser auf die Nase fällt. Dies sieht
man am vorangestellten \textbf{try}. Wir haben bereits vor diesem
Beispiel Kombinatoren gesehen. Nehmen wir uns noch einmal den lexer
vor den Parsec uns generiert hat:
\begin{lstlisting}
opStart  = oneOf "+-*/%|&=!<>"
\end{lstlisting}
\texttt{oneOf} ist hierbei eine Anweisung mit der Typsignatur:
\begin{lstlisting}
oneOf :: Stream s m Char => [Char] -> ParsecT s u m Char
\end{lstlisting}
Die Funktion akzeptiert als Argument einen String (\texttt{[Char]}) und
gibt einen Parser vom Typ Char zurück der nun irgendeinen der zuvor
angegeben Buchstaben parsed. Diese Hilfsgebilde erlauben es einem den
ansonsten nötigen ``Boilerplate'' Code zu überspringen ohne die
Lesbarkeit darunter leiden zu lassen.\cite{ParsecChar14}

Dies zeigt auch noch einmal der Input Parser, der für einen gegebenen Input:
\begin{enumerate}
\item Vorangestellten Whitespace konsumiert
\item Einen der drei Anweisungstypen parsed
\item und nur erfolgreich ist, wenn anschließend kein Input mehr übrig ist
\end{enumerate}

\begin{lstlisting}
parseInput :: Parser Anweisung
parseInput = do
  whiteSpace lexer
  a <- (parsePrint
        <|> parseZuweisung
        <|> parseFunktionsDefinition
       )
  eof
  return a
\end{lstlisting}

\noindent Dieser leicht verständliche und vor allem kompositionierbare Ansatz
macht das Parsen mit Parsec besonders anschaulich.\\
Der Input wird hierbei in eine AST ähnliche Form geparsed.

TODO: Parser Monade \cite{GHEM}

\section{Interpretieren}

Das Interpretieren der geparsten Statements ist nun lediglich Pattern
Matching gegen die in \textit{Langlang.Data} definierten Typen.
Interessant an dieser Stelle ist die Verwendung der ``Error Monade''
um sicher zu gehen, dass der Interpreter nie durch einen Fehler gestört
werden kann. Hierzu wrappt man den Unfehlbareren
Rechner in einem Fehlbaren und zieht damit die Fehlerbehandlung aus
dem Kontrollfluss des Programmes heraus.

\section{Der Rechner}

... oder warum funktionale Programmierung schön ist. Das gesamte Programm wird dann nun in \textit{langlang.hs}
zusammengeführt. Es besteht aus drei Monaden und einem Map.
\subsection{IO }

Die IO Monade ist in Haskell das Bindeglieg zwischen Programm und
Umgebung. Hier wird mit \texttt{liftIO getContents} die Nutzereingabe
gelesen und im Zusammenspiel mit der ErrorMonade die Fehlermeldungen
aus dem Programm gehoben. (Als schönes Beispiel für die Vorzüge von
laziness dient die Funktion \texttt{rechner}. Obwohl der Input durch den User
Zeile für Zeile geschieht wird direkt über den gesamten Input gemappt.)\cite{IO14}

\subsection{Error}

Die ErrorMonade hatte schon in Interpretieren ihre Erklärung aber hier
sieht man noch einmal deutlich ihre Funktionsweise. Eine Rechnung ist
entweder ein Fehler oder ein geparster Wert und anschließend entweder
eine interpretierte Anweisung oder ein Fehler. Jedoch nie eine
fehlerhafte interpretierte Anweisung.\cite{ER14}

\subsection{State}

Wer in Paradigmen der Programmierung aufgepasst hat, weiß das Werte
unveränderlich und Programmierung vorzugsweise zustandslos ist. Da in
unserem Interpreter jedoch Funktionen und Werte definierbar sein
sollen, gibt uns State die Möglichkeit einen Zustand einzuführen der
während der Ausführung von Rechner existiert, dieser jedoch nie
entkommen kann, und damit keine Auswirkung auf die aufrufende Welt
hat. Weiterhin erlaubt uns State eine Liste von Konstanten mit in den
Rechner zu geben.\cite{ST14}


\begin{abstract}
  \noindent
  \begin{itemize}
  \item \texttt{berechne} führt parsen und interpretieren für eine
    Zeile Input durch
  \item \texttt{rechner} führt \texttt{berechne} für jede Zeile Input
    aus
  \item \texttt{main} kapselt \texttt{rechner} in State und gibt einen
    Anfangszustand hinein
  \end{itemize}
\end{abstract}

\section{Fazit}

Das beschäftigen mit Haskell hat mir großen Spaß gemacht. Die
eingesetzten Technologien waren erfrischend ungewöhnlich und konnten
ein ums andere mal ins Staunen versetzten. Auf jeden Fall werde ich
die Entwicklung fortsetzen um Lang endlich den Listentyp zu geben den
es verdient hat.

\newpage
%%%%%%%%%%%%%%%%% Bibetex %%%%%%%%%%%%%%%%%%%%%%
\bibliographystyle{natdin}
\bibliography{bibliography}
%%%%%%%%%%%%%%%%% \Bibtex  %%%%%%%%%%%%%%%%%%%%%%

\end{document}
